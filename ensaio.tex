%%%%%%%%%%%%%%%%%%%%%%%%%%%%%%%%%%%%%%%%%
% Thin Sectioned Essay
% LaTeX Template
% Version 1.0 (3/8/13)
%
% This template has been downloaded from:
% http://www.LaTeXTemplates.com
%
% Original Author:
% Nicolas Diaz (nsdiaz@uc.cl) with extensive modifications by:
% Vel (vel@latextemplates.com)
%
% License:
% CC BY-NC-SA 3.0 (http://creativecommons.org/licenses/by-nc-sa/3.0/)
%
%%%%%%%%%%%%%%%%%%%%%%%%%%%%%%%%%%%%%%%%%

%----------------------------------------------------------------------------------------
%   PACKAGES AND OTHER DOCUMENT CONFIGURATIONS
%----------------------------------------------------------------------------------------

\documentclass[a4paper, 11pt]{article} % Font size (can be 10pt, 11pt or 12pt) and paper size (remove a4paper for US letter paper)

\usepackage{hyperref}
\usepackage[english,portuguese]{babel}
\usepackage[utf8]{inputenc}
\usepackage{float}

\usepackage[protrusion=true,expansion=true]{microtype} % Better typography
\usepackage{graphicx} % Required for including pictures
\usepackage{wrapfig} % Allows in-line images
\usepackage{tocloft}

\usepackage{mathpazo} % Use the Palatino font
\usepackage[T1]{fontenc} % Required for accented characters
\linespread{1.05} % Change line spacing here, Palatino benefits from a slight increase by default
\usepackage{etoolbox}
\makeatletter
\renewcommand\@biblabel[1]{\textbf{#1.}} % Change the square brackets for each bibliography item from '[1]' to '1.'
\renewcommand{\@listI}{\itemsep=0pt} % Reduce the space between items in the itemize and enumerate environments and the bibliography
\pretocmd{\chapter}{\addtocontents{toc}{\protect\addvspace{5\p@}}}{}{}
\pretocmd{\section}{\addtocontents{toc}{\protect\vspace{-4mm}}}{}{}
\renewcommand{\maketitle}{ % Customize the title - do not edit title and author name here, see the TITLE block below
\begin{flushright} % Right align
{\LARGE\@title} % Increase the font size of the title

\vspace{50pt} % Some vertical space between the title and author name

{\large\@author} % Author name
\\\@date % Date

\vspace{40pt} % Some vertical space between the author block and abstract
\end{flushright}
}

%----------------------------------------------------------------------------------------
%   TITLE
%----------------------------------------------------------------------------------------

\title{\textbf{Física antropológica e psicologia social em pesquisa crítica de redes}\\ % Title
(resumo preliminar)} % Subtitle

\author{\textsc{Renato Fabbri, Deborah Antunes, Marília Pisani} % Author
\\{\textit{IFSC/USP, PD/UFC, CCNH/UFABC}}} % Institution

\date{\today} % Date

%----------------------------------------------------------------------------------------

\begin{document}

\maketitle % Print the title section

%----------------------------------------------------------------------------------------
%   ABSTRACT AND KEYWORDS
%----------------------------------------------------------------------------------------

%\renewcommand{\abstractname}{Summary} % Uncomment to change the name of the abstract to something else


{
\selectlanguage{english}
\begin{abstract}
	translate
\end{abstract}
}




\begin{abstract}
O termo Indústria Cultural foi cunhado em 1941 pelo filósofo alemão Max Horkheimer para designar um novo modo de desenvolvimento da cultura que se faz enquanto mercadoria em um processo em que o controle social adentra a esfera da vida do indivíduo para além do mundo do trabalho e da indústria, e passa a dominar seu tempo livre. Este conceito comporta uma articulação entre subjetividade e política, que aproxima a Psicanálise de Freud e o marxismo. Amplamente divulgado na ‘Dialética do Esclarecimento’, que Horkheimer escreveu em coautoria com Theodor Adorno, o conceito de Indústria Cultural torna-se historicamente possível com os avanços tecnológicos e científicos e a decorrente produção de aparatos para o consumo do público em geral, principalmente os relativos à comunicação, como o telégrafo, o telefone, o rádio, os aparelhos de TV, as projeções cinematográficas etc. Para os autores, a mediação por esses aparatos tem consequências nos indivíduos e sua subjetividade no que tange o desenvolvimento do que chamaram de falsa consciência das massas, pensamento de ticket e potencial antidemocrático. Hoje, com o salto tecnológico digital, pode-se dizer da existência de uma Indústria Cultural 2.0 que, segundo a tese de Rodrigo Duarte, preserva quase todos os elementos de outrora capazes de formar o que se designa sujeito cativo. O objetivo desse trabalho é, através de uma análise sistemática do meio digital, compreender em que medida tais características estão preservadas e, em que medida, é possível verificar o surgimento de diferentes potencialidade não existentes anteriormente.
Embora haja facilitação dos processos massificantes e centralizadores, parece haver também uma ação com potencialidade de práxis do indivíduo, que pode melhor observar e atuar nas estruturas sociais nas quais se encontra. Para esta potencialização, que não ocorre sem contradições, apresentamos resultados em ações percolatórias no
 tecido social, conceituação de meta-sensores (sociais) e estratégias antropológicas que permitem analisar as transformações no âmbito da cultura e das subjetividades. A partir deste núcleo de conhecimento recente e pertinente, desponta uma metodologia crítica para a condução de pesquisa em sistemas complexos humanos a partir dos meios e rastros digitais, i.e. a partir de registros de interação e escrita em interfaces amistosas para o usuário final e como dados para processamento e reconhecimento de padrões e leis naturais. A crítica unifica e impulsiona ambos o rigor lógico e a flexibilidade expressiva no avanço das hipóteses a partir de teorias e dados. Permite relacionar o conteúdo com a forma e superar determinantes de sofrimento para o indivíduo e a espécie humana. Uma metodologia crítica para a pesquisa em redes utiliza e problematiza tanto os instrumentos conceituais da teoria crítica quanto coloca em tensão as metodologias de redes complexas. Esse encontro de áreas, entre a filosofia social, a física, a psicologia social e a antropologia, é sustentado pela hipótese de que o cruzamento interdisciplinar e a pesquisa empírica são centrais para a produção de conhecimento no mundo contemporâneo. Ademais, cabe frisar que esta pesquisa não é meramente elencadora de dados e conceitos pois possui, revela e opera uma intenção ética e política.
  O percurso da apresentação expressará essa intencionalidade:
  delineio do referencial teórico seguido da
   metodologia de redes
    e implicações desta articulação.
    Em especial, formas de transposição dos processos autoritários por meio da contracultura nas redes e as novas experiências de subjetividades que emergem neste contexto de "nova sensibilidade ciborgue".
\end{abstract}
% Resumo em inglês: 
% The use of digital traces of our social structures and activities is a reality for some companies and State instances. The exploitation by the individual and by Society is still incipient. This writing is a brief account of an immersion to advance this civil empowerment, beginning with experiments for collection and dissemination of information, and going through social structures streaming, resource recommendation via complex networks and natural language processing, linked data and ontological organizations of social and participatory structures.
%
%
%

\hspace*{3,6mm}\textit{Keywords:} teoria crítica, psicologia social, redes complexas, antropologia, física antropológica, pesquisa empírica, interdisciplinaridade, transdisciplinaridade

%\vspace{30pt} % Some vertical space between the abstract and first section

%----------------------------------------------------------------------------------------
%   ESSAY BODY
%----------------------------------------------------------------------------------------
%\newpage
%\tableofcontents


%\bibliographystyle{unsrt}
\bibliographystyle{plain}
\bibliography{ensaio}

%----------------------------------------------------------------------------------------

\end{document}
